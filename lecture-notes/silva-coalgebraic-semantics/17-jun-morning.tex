\documentclass{article}
\usepackage[utf8]{inputenc}
\usepackage{amsmath, amssymb, amsthm, bussproofs, mathtools}

\title{Coalgebraic Semantics Lecture 1}
\author{Farzaneh Derakhshan, Tao Gu, Aditya Oak}
\date{17 June, 2019}


%%%%%%%%%% new commands
\newcommand{\alphA}{\mathcal{A}}
\newcommand{\ocaml}{\mathsf{OCaml}}
\newcommand{\cons}{\mathsf{cons}}
\newcommand{\natN}{\mathbb{N}}
\newcommand{\len}{\mathsf{len}}
\newcommand{\odd}{\mathsf{odd}}
\newcommand{\even}{\mathsf{even}}
\newcommand{\merge}{\mathsf{merge}}
\newcommand{\head}{\mathsf{head}}
\newcommand{\tail}{\mathsf{tail}}
% \newcommand{\even}{\mathsf{even}}
% \newcommand{\odd}{\mathsf{odd}}
\newcommand{\bis}{\sim}
\newcommand{\imply}{\Rightarrow}

%%%%%%%%%% new environments
% \theoremstyle{theorem}
\newtheorem{theorem}{Theorem}
\newtheorem{lemma}[theorem]{Lemma}
\newtheorem{proposition}[theorem]{Proposition}
\theoremstyle{definition}
\newtheorem{definition}[theorem]{Definition}
\theoremstyle{definition}
\newtheorem{example}[theorem]{Example}


\begin{document}
\maketitle

This lecture is a first introduction to coalgebra. We first recall the inductive data types and the principle of induction. Then we introduce some basic coinductive data types and the principle of coinduction.

\section{Finite and infinite data types}
Throughout this note we fix an arbitrary alphabet set $\alphA$. The elements in $\alphA$ are called letters.
\begin{example}
Our first example is the inductive data structure $\alphA^*$ consisting of all finite words on $\alphA$. There are several ways of describing $\alphA^*$:
\begin{itemize}
    \item In $\ocaml$, one can write:
    \[
    \textbf{type list a = nil $\mid$ cons a l }
    \]
    \item By derivation rules:
    \begin{center}
        \begin{tabular}{cc}
            \bottomAlignProof
                \AxiomC{}
                \UnaryInfC{$\epsilon \in \alphA^*$}
                \DisplayProof
            &
            \bottomAlignProof
                \AxiomC{$a\in \alphA$}
                \AxiomC{$u\in \alphA^*$}
                \BinaryInfC{$a \cdot u \in \alphA^*$}
                \DisplayProof
        \end{tabular}
    \end{center}
\end{itemize}
\end{example}
Given a letter in $\alphA$ and a word in $\alphA^*$, we can {\it construct} a new word in $\alphA^*$ by appending the letter to the word. The above definition does not only define $\alphA^*$ using its constructors, but also provides a way to inductively define functions operating on $\alphA^*$. In the following Example \ref{ex:leng} we use constructors of $\alphA^*$ to define the function $\len$.
\begin{example}\label{ex:leng}
We define function $\len \colon \alphA \to \natN$ inductively using the constructors of $\alphA^*$:
$$\len(\epsilon)=0, \quad  \quad \quad \len(a\cdot u)= 1+\len(u).$$
Intuitively $\len$ simply counts the length of a finite word. By the inductive construction of finite words, it is enough to explain the behaviour of the function (i) on the empty word, and (ii) on $a \cdot u$ assuming its behaviour on $u$.
\end{example}

\begin{example}\label{ex:concat}
Concatenation function $;\ \colon \alphA^* \times \alphA^* \to \alphA^*$ is defined by structural induction on its first argument as:
$$\epsilon;u=u, \quad  \quad \quad (a\cdot u); v= a \cdot (u;v).$$
Similar to the previous example, we used recursive call to define concatenation.
\end{example}
We can also prove properties of functions defined on inductive types using their constructors:
\begin{example}
We want to prove $\len(u;v)= \len(u)+\len(v)$ given the definitions of $\len$ and $;$ functions in examples \ref{ex:leng} and \ref{ex:concat}. The proof goes by induction on the first argument:
\begin{description}
    \item[Base case.] $\len(\epsilon;v) = \len(v) = 0+\len(v) = \len(\epsilon)+\len(v)$
    \item[Inductive case.] $\len(a \cdot u;v)=\len(a\cdot(u;v))=1+\len(u;v)$ by definitions of $\len$ and concatenation.
By inductive hypothesis, $\len(u;v) = \len(u)+\len(v)$. Thus $\len(a \cdot u;v)= 1+\len(u;v) = 1 + \len(u)+\len(v) = \len(a\cdot u)+\len(v)$.
\end{description}
\end{example}
Inductive datatypes come with an inductive principle that works on their constructors. For the coinductive infinite data types we have a similar situation. But instead of defining the values of constructors, for coinductive types we define the value of {\it deconstructors} in a dual manner. 

\section{Infinite Sequences over $\alphA$}
The set $\alphA^\omega$ of all infinite sequences over $\alphA$ is defined as $\{\sigma \mid \sigma \colon \natN\to \alphA \}$. We use the notation $\sigma = (\sigma_0, \sigma_1, \dots ) $. We define deconstructing functions
\begin{align*}
   \head(\sigma) & \coloneqq \sigma(0) \\
   \tail(\sigma)(n) & \coloneqq \sigma(n+1)
\end{align*}
 
\textbf{Coalgebraic semantics:} It can be considered as a way to reason about the programs that operate on infinite data structures. More details on this will be provided in the following lectures.
 
\subsection{Functions over $\alphA^{\omega}$}
We can define some basic operations on $\alphA^\omega$:
\begin{itemize}
\item $ \even\colon \alphA^{\omega} \to \alphA^{\omega} $,
\[
    \even \colon (\sigma_0, \sigma_1, \dots) \mapsto (\sigma_0, \sigma_2 , \sigma_4, \dots) 
\]
\item $\odd \colon \alphA^{\omega} \to \alphA^{\omega} $
\[
    \odd \colon (\sigma_0, \sigma_1, \dots) \mapsto (\sigma_1, \sigma_3 , \sigma_5, \dots) 
\]
\item $ \merge \colon \alphA^{\omega} \times \alphA^\omega \to \alphA^{\omega} $
\[ 
    \merge \colon (\sigma_0, \sigma_1 , \dots) , (\tau_0, \tau_1, \dots) \mapsto (\sigma_0, \tau_0 , \sigma_1 , \tau_1, \dots) 
\]
 
\end{itemize} 
 
Recursive definitions of the above given functions:
\begin{itemize}
\item 
$ 
    \begin{cases}
        \head (\merge (\sigma, \tau)) = \head (\sigma) \\
        \tail (\merge (\sigma, \tau)) = \merge(\tau, \tail(\sigma)))
    \end{cases}
$
\item 
$
    \begin{cases}
        \head(\even(\sigma)) = \head(\sigma) \\
        \tail(\even(\sigma)) = \even(\tail(\tail(\sigma))) = \odd(\tail(\sigma))
    \end{cases}
$
\item 
$
    \begin{cases}
        \head(\odd(\sigma)) = \head(\tail(\sigma))
 \\
        \tail(\odd(\sigma)) = \tail(\even(\tail(\sigma)))
    \end{cases}
$
\end{itemize}

\begin{proposition}
$ \merge(\even(\sigma), \odd(\sigma)) = \sigma $
\end{proposition}
\begin{proof}
Let $\rho = \merge(\even(\sigma), \odd(\sigma))$.
Now we show that $\forall n \in \natN, \rho(n) = \sigma(n) $:
\begin{description}
    \item[Base case.] 
    \begin{align*}
        \rho(0) & = \head(\merge(\even(\sigma), \odd(\sigma))) \\
        & = \head(\even(\sigma)) \\
        & = \head(\sigma) \\
        & = \sigma(0)
    \end{align*}
    \item[Inductive case.]
     \begin{align*}
        \rho(n+1) & =  \merge(\odd(\sigma), \tail(\even(\sigma)))(n) \\
        & =  \merge(\odd(\sigma), \odd(\tail(\sigma)))(n) \\
        & = \merge( \even(\tail(\sigma)), \odd(\tail(\sigma)) )(n) & (\text{IH}) \\
        & = \tail(\sigma)(n) \\
        & = \sigma(n+1)
    \end{align*}
\end{description}
    
\end{proof}

\begin{proposition}
$ \even(\merge(\sigma, \tau)) = \sigma $
\end{proposition}
\begin{proof}
Let $\rho = \even(\merge(\sigma, \tau))$.
\begin{description}
    \item[Base case.] 
    \begin{align*}
        \rho(0) & = \merge(\sigma, \tau)(0) \\
        & = \sigma(0) 
    \end{align*}
    \item[Inductive case.]
     \begin{align*}
        \rho(n+1) & =  \merge(\sigma, \tau)(2(n + 1)) \\
        & =  \sigma(n+1) 
    \end{align*}
\end{description}
\end{proof}

Similarly we can show    $\odd(\merge(\sigma, \tau)) = \tau$.
    
\section{Bisimulation and Coinduction}
Proof by induction is common and useful on inductive data structures. Similarly, one would expect coinductive proof on coinductive data structures. One convenient way for this is to find a bisimulation, and then apply the coinductive principle. To get the feeling of coinduction, we will focus on infinite streams $\alphA^*$ here.
\begin{definition}\label{def:bis-stream}
A relation $R \subseteq \alphA^* \times \alphA^*$ is a \emph{bisimulation} if $\forall (\sigma, \tau) \in R$, 
\begin{enumerate}
    \item $\head(\sigma) = \head(\tau)$
    \item $(\tail(\sigma), \tail(\sigma)) \in R$
\end{enumerate}
\end{definition}

We say that two infinite sequences $\sigma$ and $\tau$ are \emph{bisimilar}, denoted as $\sigma\bis \tau$, if there exists some bisimulation between them. Bisimulation provides a way of proving by coinduction:
\begin{lemma}[Coinductive Principle for $\alphA^*$]
For all $\sigma, \tau\in \alphA^*$,
\begin{equation*}
    \sigma\bis \tau \imply \sigma = \tau
\end{equation*}
\end{lemma}
\begin{proof}
Suppose $R\subseteq \alphA^* \times \alphA^*$ is a bisimulation such that $(\sigma, \tau) \in R$. It suffices to show that $\sigma(n) = \tau(n)$, for all $n\in \natN$. We prove by induction on $n$.
\begin{itemize}
    \item For $n=0$, this is exactly $(1)$ in definition \ref{def:bis-stream}.
    \item For $n + 1$, note that by $(2)$ in definition \ref{def:bis-stream} we have $\tail(\sigma) \bis \tail(\tau)$. So we have:
    \begin{align*}
        \sigma(n+1) & = \tail(\sigma)(n) \\
        & = \tail(\tau)(n) & (\text{IH}) \\
        & = \tau(n+1)
    \end{align*}
\end{itemize}
Therefore $\sigma = \tau$.
\end{proof}

In some cases, we may not have a bisimulation at first sight. It is common that we start from the conclusion we want to proof, and construct a bisimulation by adding the necessary pairs.
\begin{example}
Consider the equation
    \begin{equation}\label{eq:merge-id}
        \merge(\even(\sigma), \odd(\sigma)) = \sigma 
    \end{equation}
To derive \eqref{eq:merge-id} using the coinduction principle, we would like to construct some bisimulation $R$ which contains the pair consisting of the streams on both sides of \eqref{eq:merge-id}. Consider the relation 
    \[
    R_0 = \{ (\merge(\even(\sigma), \odd(\sigma)), \sigma) \mid \sigma\in \alphA^\omega \}
    \]
To ensure condition $(2)$ in definition \ref{def:bis-stream}, we need to add the following pairs to the relation:
    \[
    R_1 = \{ (\tail(\merge(\even(\sigma), \odd(\sigma))), \tail(\sigma)) \mid \sigma\in \alphA^\omega \}
    \]
Proceeding like this, we get a sequence of relations, and the union of all these relations should give a bisimulation. Fortunately we do not need to add a lot, since $R_1$ is already included in $R_0$:
\begin{align*}
    \tail(\merge (\even(\sigma), \odd(\sigma)) ) & = \merge( \odd(\sigma), \tail(\even(\sigma)) ) \\
    & = \merge( \even (\tail(\sigma)), \odd(\tail(\sigma)) )
\end{align*}
For any $\sigma$, consider $\tau = \tail(\sigma)$, and we know that $(\merge(\even(\tau),\odd(\tau)), \tau )\in R_0$.
\end{example}

\begin{example}
Consider the equation
\begin{equation}
    \even(\merge (\sigma, \tau)) = \sigma
\end{equation}
Following the above strategy, we construct a sequence of relations:
\begin{align*}
    R_0 & = \{ (\even (\merge(\sigma,\tau)), \sigma) \mid \sigma, \tau \in \alphA^\omega \} \\
    R_1 & = \{ ( \tail(\even(\merge(\sigma,\tau))), \tail(\sigma) ) \mid \sigma, \tau \in \alphA^\omega \} \\
    R_2 & = \{ ( \tail(\tail(\even(\merge(\sigma,\tau)))), \tail(\tail(\sigma)) ) \mid \sigma, \tau \in \alphA^\omega \} \\
    \cdots
\end{align*}
Note that
\begin{align*}
    \tail(\even(\merge(\sigma,\tau))) & = \odd( \tail(\merge(\sigma,\tau)) ) \\
    & = \odd( \merge(\tau, \tail(\sigma)) ) \\
    \tail(\odd(\merge(\sigma,\tau))) & = \even( \tail(\merge(\sigma,\tau)) ) \\
    & = \even(\merge(\tau, \tail(\sigma)))
\end{align*}
Then it is easy to see that $R_0 \supseteq R_2 \supseteq \cdots$ and $R_1 \supseteq R_3 \supseteq \cdots$. So the union of the whole sequence terminates as $R \coloneqq R_0\cup R_1$. $R$ is then a bisimulation containing all $(\even(\merge(\sigma, \tau)), \sigma)$.
\end{example}

\section*{Homework}
Suppose $R, S\subseteq \alphA^\omega \times \alphA^\omega$ are bisimulations.
\begin{enumerate}
    \item $R; S$ is bisimulation, where $(x,z) \in R;S$ iff $\exists y$ such that $(x,y)\in R$ and $(y,z) \in S$.
    \item $R^{o}$ is bisimulation, where $(x,y) \in R^{o}$ iff $(y,x)\in R$.
\end{enumerate}
\end{document}
